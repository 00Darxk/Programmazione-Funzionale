\documentclass{article}

\usepackage{cancel}
\usepackage{amsmath,amssymb}
\usepackage[includehead,nomarginpar]{geometry}
\usepackage{graphicx} \usepackage{amsfonts} 
\usepackage{verbatim}
\usepackage{mathrsfs}  
\usepackage{lmodern}
\usepackage{braket}
\usepackage{bookmark}
\usepackage{fancyhdr}
\usepackage{romanbarpagenumber}
\usepackage{minted}
%\usepackage{subfig}
\usepackage[italian]{babel}
\usepackage{float}
%\usepackage{wrapfig}
%\usepackage[export]{adjustbox}
\usepackage{contour}
\usepackage[normalem]{ulem}
\allowdisplaybreaks

\setlength{\headheight}{12.0pt}
\addtolength{\topmargin}{-12.0pt}
\graphicspath{ {./Immagini/} }

\hypersetup{
    colorlinks=true,
    linkcolor=black,
    pdftitle={Esercizi Svolti di Programmazione Funzionale},
    pdfauthor={Giacomo Sturm},
    pdfsubject={Programmazione Funzionale, OCaml, ML},
    pdfkeywords={Programmazione Funzionale, OCaml, ML}
}

\newsavebox{\tempbox} %{\raisebox{\dimexpr.5\ht\tempbox-.5\height\relax}}


\makeatother
\renewcommand{\contentsname}{Indice}
\numberwithin{equation}{subsection}
\newcommand{\tageq}{\tag{\stepcounter{equation}\theequation}}
\AtBeginDocument{%
    \renewcommand{\figurename}{Fig.}
}
\renewcommand{\ULdepth}{1.8pt}
\contourlength{0.6pt}
\newcommand{\myuline}[1]{%
    \uline{\phantom{#1}}%
    \llap{\contour{white}{#1}}%
}
\fancypagestyle{link}{\fancyhf{}\renewcommand{\headrulewidth}{0pt}\fancyfoot[C]{Sorgente del file \LaTeX disponibile al seguente link: \url{https://github.com/00Darxk/Programmazione-Funzionale/}}}

\begin{document}

\title{%
    \textbf{Programmazione Funzionale}  \\ 
    \large Esercizi Svolti di Programmazione Funzionale \\
    \textit{Anno Accademico: 2024/25}}
\author{\textit{Giacomo Sturm}}
\date{\textit{Dipartimento di Ingegneria Civile, Informatica e delle Tecnologie Aeronautiche \\
Università degli Studi ``Roma Tre"}}

\maketitle
\thispagestyle{link}

\clearpage


\pagestyle{fancy}
\fancyhead{}\fancyfoot{}
\fancyhead[C]{\textit{Programmazione Funzionale - Università degli Studi ``Roma Tre"}}
\fancyfoot[C]{\thepage}
\pagenumbering{Roman}

\tableofcontents

\clearpage
\pagenumbering{arabic}


\section{Esercitazione 1}
%% TODO add esercitazione n. 1

\clearpage

\section{Esercitazione 2}
Definire una funzione \verb|ultime_cifre:| \verb|-> int * int| che riporti il valore intero delle due ultime cifre di un \verb|int|. Dato un numero, il suo modulo base 10 restituisce l'ultima cifra. Mentre per ottenere la penultima si effettua una divisione intera per 10 per eliminare l'ultima cifra e si riapplica il modulo base 10 per ottenere questa cifra:
\begin{minted}{ocaml}
    let ultime_cifre x = (abs(x) mod 10, abs(x/10) mod 10);;
\end{minted}

%% TODO esercitazione 17/3/25

Una cifra è bella se è 0, 3 o 7; un numero è bello se la sua ultima cifra è bella e la penultima (se esiste) non lo è. Definire un predicato \verb|bello: int -> bool|, che determini se un numero è bello, la funzione non deve mai sollevare eccezioni, ma riportare sempre un \verb|bool|. 
\begin{minted}{ocaml}
    let bello x = 
        match abs(x mod 10) with
            0 | 3 | 7 -> (match (abs(x mod 10))/10 with
                0 | 3 | 7 -> x < 10
                    3 | 7 -> false
                    | 0 -> x < 100
                    | _ -> true)
            | _ -> false;;
\end{minted}

Scrivere una funzione \verb|data: int * string -> bool|, che applicata ad una coppia \verb|(d, m)| di un intero \verb|d| e una stringa \verb|m|, determini se la coppia rappresenta una data corretta, assumendo che l'anno non sia bisestile, si assuma che i mesi siano rappresentati da stringhe con caratteri minuscoli. La funzione on deve sollevare eccezioni, ma riportare sempre un \verb|bool|. 
È utile scrivere una funzione \verb|gdm| per determinare i giorni di un mese:
\begin{minted}{ocaml}
    let gdm m = 
        match with
        "gennaio"  | "marzo" | "maggio" | "luglio" | "agosto" | "ottobre" | "dicembre" -> 31
        | "aprile" | "giugno" | "settembre" | "novembre" -> 30
        | "febbraio" -> 28
        | _ -> 0
\end{minted}
La funzione principale è quindi:
\begin{minted}{ocaml}
    let data (d,m) = d <= gdm m && d > 0;;
\end{minted}

\clearpage

\section{Esercitazione 3}

\begin{minted}{ocaml}
    exception NoElement ;; 
    let maxlist = function
          [] -> raise NoElement
        | x::xs -> let rec aux m = function
              [] -> m
            | x::xs -> if x > m then aux x xs
                       else aux m xs
            in aux x xs
\end{minted}

Si ipotizza di non avere a disposizione l'operatore \verb|@|, definire la concatenazione:
\begin{minted}{ocaml}
    let rec append l1 l2 = match l1 with
          [] -> l2
        | x::xs -> x::(append xs l2) 
\end{minted}
In modo iterativo diventa:
\begin{minted}{ocaml}
    let append l1 l2 =
        let rec aux a = function
              [] -> a
            | x::xs -> aux (x::a) xs
        in aux l2 List.rev(l1)
\end{minted}


\begin{minted}{ocaml}
    let rec nth n = function
          [] -> raise NoElement
        | x::xs -> if n < 0 then raise NoElement
                   else if n = 0 then x
                        else nth (n-1) xs;;
\end{minted}


\begin{minted}{ocaml}
    let rec nondec = function
          [] | [a] -> true
        | x::y::xs -> if x > y then false
                      else nondec (y::xs)
\end{minted}


%% TODO idfk 

\begin{minted}{ocaml}
    let min_dei_max ls = 
        let rec listmax acc = function
              [] -> acc
            | x::xs -> listmax ((maxlist x)::acc) xs
        in let rec minlist = function
              [] -> raise NoElement
            | x::xs -> try let y = minlist(xs)
                           in min x y
                       with _ -> x
        in minlist(listmax [] ls);;
\end{minted}
%% TESTING funzione con ocaml repl
% min_dei_max [[1;2;3];[3;4;1;0];[78]];;

Concatena l'accumulatore con i massimi di ciascuna lista del secondo argomento, l'argomento implicito è una lista di liste, essendo l'argomento della funzione esterna \verb|ls|. Si può inferire il tipo dato che ad un elemento della lista \verb|x| viene passato come argomento a \verb|maxlist|, quindi deve essere una lista. 



\verb|split2| agisce in modo simile alla funzione \verb|split| definita precedentemente, divide in due parti ...

\clearpage

\section{Esercitazione 4}

\clearpage

\section{Esercitazione 5}

\clearpage

\section{Esercitazione 6}

Si definisca la funzione \verb|find|, definita nel modulo List, tale che \verb|find p lst| restituisca il primo elemento della lista \verb|lst| che soddisfa il predicato \verb|p|, altrimenti solleva un'eccezione. 
\begin{minted}{ocaml}
    exception NotFound;;
    let rec find p = function
        | [] -> raise NotFound
        | x::xs -> if (p x) then x
                   else find p xs
\end{minted}

Si definisce la funzione \verb|takeawhile p lst| che riporti la più lunga parte iniziale di \verb|lst| costituita da tutti gli elementi che soddisfano il predicato \verb|p|. 
\begin{minted}{ocaml}
    let rec takeawhile p = function
        | [] -> []
        | x::xs -> if (p x) then x::(takeawhile p xs)
                   else []
\end{minted}
Definire ora un predicato \verb|p|, tale che la funzione \verb|takeawhile| restituisca la parte iniziale contenente solo numeri pari:
\begin{minted}{ocaml}
    takeawhile (function x -> x mod 2 = 0) l    
\end{minted}

Si definisca la funzione \verb|partition: ('a| $\rightarrow$ \verb|bool)| $\rightarrow$ \verb|'a list| $\rightarrow$ \verb|('a list * 'a list)|, tale che applicata ad un predicato \verb|p| ed una lista \verb|lst| restituisca una coppia di liste, dove la prima contiene tutti gli elementi che soddisfano il predicato, mentre la seconda lista contiene tutti gli elementi che non lo soddisfano:
\begin{minted}{ocaml}
    let partition p l =
        let rec aux (yes, no) = function
            | [] -> (yes, no)
            | x::xs -> if (p x) then aux ((yes@[x]), no) xs
                       else aux (yes, (no@[x])) xs
        in aux ([], []) l    
\end{minted}

Definire la funzione \verb|pairwith|, che dato un elemento ed una lista restituisce una lista di coppie formate dall'elemento passato e l'elemento corrispondente della lista passata, utilizzando \verb|List.map|
\begin{minted}{ocaml}
    let pairwith x l = List.map (function y -> (x, y) ) l
\end{minted}

Definire la funzione \verb|setdiff| per la differenza insiemistica, utilizzando \verb|List.filter|:
\begin{minted}{ocaml}
    let rec isin l x = match l with
        | [] -> true
        | y::ys -> if (x = y) then false
                   else isin ys x;;
    let setdiff l1 l2 = List.filter (isin l2) l1
\end{minted}
In maniera più concisa, invece di utilizzare un'altra funzione \verb|isin| si può utilizzare la funzione \verb|exists| dal modulo List, che prende un predicato ad una lista e restituisce un booleano che rappresenta se esiste un elemento nella lista che verifica quel predicato. 
\begin{minted}{ocaml}
    let mem l x = List.exists ((=) x) l;;
    let setdiff l1 l2 = List.filter (not (mem l2)) l1
\end{minted}

Un'altra soluzione possibile è la seguente:
\begin{minted}{ocaml}
    let rec setdiff l1 = function
        | [] -> l1
        | x::xs -> setdiff (List.filter (function y -> y <> x) l1) xs    
\end{minted}


Determinare la funzione \verb|verifica_matrice: int| $\rightarrow$ \verb|int list list| $\rightarrow$ \verb|bool|, che dato un intero ed una matrice rappresentata come liste di liste, controlla se è presente almeno una riga dove tutti gli elementi sono minori di $n$. Si può definire una funzione \verb|verifica_riga| che controlla la condizione su un'unica riga:
\begin{minted}{ocaml}
    let verifica_riga n l = List.for_all ((>) n) l 
\end{minted}
La funzione di partenza si realizza quindi come:
\begin{minted}{ocaml}
    let verifica_matrice n l = List.exists (verifica_riga n) l    
\end{minted}

\begin{minted}{ocaml}
    let rec tutte_liste_con n x y = match n with
        | 0 -> [[]]
        | n -> let r = tutte_liste_con (n - 1) x y
               in (List.map (List.cons x) r)@(List.map (List.cons y) r)
\end{minted}

\begin{minted}{ocaml}
    let rec interleave n = function
        | [] -> [[n]]
        | x::xs -> (n::x::xs)::(List.map ((@) [x]) (interleave n xs))
\end{minted}

% lista di tutte le possibili permutazioni
\begin{minted}{ocaml}
    let rec permut l = match l with
        | [] -> [[]]
        | x::xs -> List.flatten (List.map (interleave x) (permut xs))
\end{minted}


\clearpage

\end{document}